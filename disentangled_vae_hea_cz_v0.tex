\documentclass[a4paper]{article}

%% Language and font encodings
\usepackage[T1]{fontenc}
\usepackage[utf8x]{inputenc}
\usepackage[english]{babel}

\usepackage[colorlinks=true, allcolors=blue]{hyperref}

\urlstyle{tt}
\newcommand{\email}[1]{\href{mailto:#1}{\tt{\nolinkurl{#1}}}}
\newcommand{\orcid}[1]{ORCID: \href{https://orcid.org/#1}{\tt{\nolinkurl{#1}}}}

\usepackage[sfdefault,lf]{carlito}
%% The 'lf' option for lining figures
%% The 'sfdefault' option to make the base font sans serif
\usepackage[parfill]{parskip}
\renewcommand*\oldstylenums[1]{\carlitoOsF #1}
\usepackage{fancyhdr}
\usepackage{natbib}
\usepackage{authblk}
\setlength{\headheight}{41pt}

%% Sets page size and margins
\usepackage[a4paper,top=3cm,bottom=2cm,left=3cm,right=3cm,marginparwidth=1.75cm]{geometry}

%% Useful packages
\usepackage{amsmath}
\usepackage{graphicx}
\usepackage{booktabs}

\usepackage[colorinlistoftodos]{todonotes}

\newcommand{\cz}[1]{\hl{\textbf{cz:} #1}}



\cz{I removed the banner header as we usually submit Materials design work to arXiv rather than EngrXiv.}
%\renewcommand{\headrulewidth}{0pt}
% \fancyhead[L]{Posted: \today}
% \fancyhead[R]{
% \includegraphics[width=4cm]{./figures/engrXiv_banner.png}
% }
\pagestyle{plain}
\title{Data-efficient and interpretable inverse materials design using a disentangled variational autoencoder}
\author[1, $\dag$, *]{Zulqarnain Khan}
\author[1, 2, $\dag$, *]{Cheng Zeng}
\author[2]{Nathan L. Post}
\affil[1]{Institute for Experiential AI, Northeastern University, Boston, MA 02115, United States}
\affil[2]{The Roux Institute, Northeastern University, Portland, ME 04101, United States}
\affil[$\dag$]{These authors contribute equally: Zulqarnain Khan, Cheng Zeng.}
\affil[*]{Corresponding author: \email{z.khan@northeastern.edu}, \email{c.zeng@northeastern.edu}}
\date{}
\cz{Added the author info, but we can discuss...}

\begin{document}
\maketitle
\thispagestyle{fancy}

\begin{abstract}
Your abstract.
\end{abstract}

\section{Introduction}

%%%% Materials importance and Materials design history
Materials play a pivotal role in shaping the modern society and many grand technology challenges are in essence materials challenges, ranging from lower-cost battery materials for energy storage, to quantum computing materials and bio-compatible materials for healthcare applications~\cite{}.
Thanks to advances in high-throughput computing~\cite{}, robotics~\cite{}, machine learning force fields~\cite{} and materials open datasets~\cite{}, materials design and discovery have now reached an unprecedented rate and scale~\cite{}.
Although advances in algorithm and hardware significantly reduce the computation time for each iteration, it can still take extensive iterations to locate a number of materials candidates with desired properties~\cite{}.

%%%% Inverse materials design, its state-of-the-art, limitations
Inverse materials design unlock the potential to identify new materials using the target property as the input~\cite{}.
However, current inverse materials design primarily uses a generative model with an unsupervised learning flavor to learn a latent compact representation of materials, and it assumes that the learned latent space will implicitly entangle the relationship between materials representations and target properties~\cite{}.
It is not ideal as one will need to perform post-optimization to explore materials with better target properties and may even fail to find any useful materials.
A recent work by Xie and Tomioka et al introduced diffusion-based generative processes together with a fine tuning process to discover materials with multiple target properties such as magnetic property and supply chain risk~\cite{}.
Although a state-of-the-art discovery rate and materials stability were reported, the full periodicity of the crystal structures restricts the design space for inorganic materials, the complexity of large datasets and diffusion models hinder a wider application and the interpretability of the methods, and the practicality of the new materials discovered in this process is doubtful without an expert insight~\cite{}.

%%% The need and summary of our work
Therefore, there is an urgent need to create a workflow for inverse materials design that is data efficient, interpretable and can be geared towards multi-property optimization.
Here we introduced a disentangled generative model using a semi-supervised variational autoencoder for the inverse design of complex materials.

\section{Proposed Model}

\section{Results and Discussion}

\section{Conclusion and Outlook}


\bibliographystyle{bibstyle}
\bibliography{refs}

\end{document}